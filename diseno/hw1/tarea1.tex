\documentclass[aps,prl,reprint,rmp]{revtex4-1}
\usepackage[utf8]{inputenc}
\usepackage{gensymb}
\usepackage{float}
\usepackage[spanish]{babel}
\usepackage{graphicx}
\usepackage{babel}
\usepackage{amsmath}

\begin{document}
\title{Propuesta de Proyecto Final}


\author{Juan Sebastian Vargas}
\affiliation{Universidad de los Andes, Departamento de f\'isica}
 

\date{18 de febrero de 2016}

\setlength{\columnsep}{1cm}






\maketitle



\section{Punto 1}


\subsection{1a}
GCL(Guarded Command Language) es un lenguaje en el cual se guarda un “comando” o mas exactamente una proposición que se debe cumplir antes de que la instrucción se ejecute. Cuando se ejecuta la instrucción se puede suponer que esta proposición es verdadera, si esta proposición es falsa no se ejecutara la instrucción.\\

Con esto se genera que de la misma manera que se puede probar el correcto funcionamiento de un autómata, se puede demostrar que un programa escrito con GCL es correcto o no.\\ 

Del mismo modo que la maquina de Turing solo puede trabajar en funciones computables, un computador personal también, pero para ver la similitud toca ver el computador personal de una manera mas básica, el procesador de un computador personal solo trabaja con unas cuantas operaciones básicas, las operaciones que realiza un computador es la composición de operaciones básicas. De este modo el computador también funciona sobre funciones computables.




 
\subsection{1.b}

No, lo único que haría esta nueva instrucción es resumir un par de las básicas, el lenguaje no seria capas de hacer algo que no se pudiera hacer antes, la única diferencia es que se podría compactar las cosas.

\section{Punto 2}




\section{Punto 3}




\section{Punto 4}

\subsection{4.1}

\subsection{4.2}

\subsection{4.3}




\section{Punto 5}



\end{document}
